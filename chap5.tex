\chapter{Conclusions, limitations and future work}
\label{conclusion}

SPF2 is the new major release of SPF. In this document, I explained the main changes made to this software. In particular, I updated the entire project to Android Studio, moving dependencies into Maven repositories. Also, I changed the theme to Material Design, using support libraries and other third party softwares. SPF2 is fully compatibile with Android Lollipop (5.x) and Marshamallow (6.0). I removed \emph{AllJoyn} and improved the Wi-Fi Direct Middleware to be able to manage group of devices.

However, there some open issues and improvements that you can do to SPF. To be able to extend this software in an interesting way, you should completely redesign the communication between SPFApp and its submodules. If you'll move SPFApp into another project and you'll implement a remote communication between SPFApp and the framework exposing methods, you will really able to extends SPF to more interesting scenarios.

Also, the first implementation of SPF completely ignored the different roles of nodes into a network (like Master/Slave or Group Owner/Client). For timing constraints, I chose a quick solution to archive this, adding switches into the GUI and send their states to the middleware. This solution is temporary, because it is terrible from the design point of view. If you have enough time, you should really extract \textsf{SPFApp} from the main Android Studio's project. I mean that, \textsf{SPFFramework} (with \textsf{SPFShared}, \textsf{SPFLib} and \textsf{SPFWFDMiddleware}) should be a library available in Maven, and demo applications could import this framework using a dependency into their \textsf{build.gradle}'s files. \textsf{SPFApp} should be an optional app to configure \textsf{SPFFramework} using remote communication. In this way, you can simply reuse the same framework for GOs and Clients, for example Shop manager and his customers, without to release on the market an application that can creates other GOs. Also, in this way you can remove the terrible workaround that I obliged to implement to build this new major release.

I repeat, the main focus for the next versions should be the extraction of \textsf{SPFApp} from the main project, creating a standalone project with remote calls to the previous one.



\section*{Open issues}
Obviously, there are other open issues. I decided to write all known problems here.
\begin{enumerate}
	\item If discovery phase fails, create a temporized system to restart it automatically.
	\item When SPF enters in \textsf{onConnectionInfoAvailable()}, start a timer on the client. At the end of the countdown, if this device isn't connected to the GO it should restart the discovery phase.
	\item Refactor \textsf{MainActivity.java} to reduce codelines.
	\item Updates the proximity switch in the GUI accordingly to the service status. For example, when Eternal Connect is working, you should update the switch. I suggest to use \emph{Broadcast receivers} (not Local, but remote declared into the \textsf{AndroidManifest.xml}). Use \textsf{GroupInfosFragment} as an example.
	\item Remove all \emph{ListViews} and replace they with \emph{RecyclerView}. Use \textsf{GroupInfosFragment} as an example.
	\item Fix Tab positioning for tablets/smartphones. You should find a way to display centered tabs that fill the entire screen width, but with text on a single line.
	\item Improve the \emph{standard mode} into the Wi-Fi Direct Middleware.
	\item Replace \emph{materialtabstrip library} with the official version included into \emph{Design support library} by Google, either for \textsf{CouponingProvider} or \textsf{CouponingClient}.
	\item Replace \emph{multiselection library} in \textsf{CouponingProvider} and \textsf{CouponingClient} with an official version.
	\item Update the \emph{profile identifier} accordingly to the device role into the network (GO or Client). Remember that I'm using the ``AP" prefix to identify GOs.
	\item Check if Wi-Fi is enabled during start-up.
	\item Rewrite the entire \textsf{ProfileFragment} GUI using \emph{RecyclerView} an statical xml's layout to simplify the code.
\end{enumerate}



