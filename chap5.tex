\chapter{Conclusions, limitations and future work}
\label{conclusion}

SPF2 is the new major release of SPF. In this document i explained the main changes made to this software. In particular I updated the entire project to Android Studio, moving dependencies into maven repositories. Also, I changed the theme to Material Design, using support libraries and other third party softwares. SPF2 is fully compatibile with Android Lollipop (5.x) and Marshamallow (6.0). I removed AllJoyn and improved the Wi-Fi Direct middleware to be able to manage group of devices.


However, there some open issues and improvements that you can do to SPF.

To be able to extend this software in an interesting way, you should completely redesign the communication between SPFApp and its sub-modules. If you'll move SPFApp into another project and you'll implement a remote communication between SPFApp and the framework exposing methods, you will really able to extends SPF to more interesting scenarios.

The first implementation of SPF completely ignored the different node's role into a network (like Master/Slave or Group Owner/Client). For time problems, I chose a quick solution to archive this, adding switches into the GUI and send their states to the middleware. This solution is temporary, because it is terrible from the design point of view. If you have time you should really extract SPFApp from the main Android Studio's project. I mean that SPFFramework (with SPFShared, SPFLib and SPFWFDMiddleware) should be a library available in Maven, and demo applciations could import this framework via compile dependency into their build.gradle's files. SPFApp should be an optional app to configure SPFFramework using remote communication. In this way you can simply reuse the same framework for GOs and Clients, for example Shop Manager and his customers, without to release on the market an application that can become a GO. Also, in this way you can remove all terrible workaround that i was oblied to implement to build this new major release.