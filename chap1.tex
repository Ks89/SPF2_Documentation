\chapter{Introduction}
\label{Introduction}

This document describes SPF2 (Social Proximity Framework 2), the new major release of the software written for the master thesis available at http://hdl.handle.net/10589/106727. Before you could understand this document you should really read the first one.

The main objectives of this second major release are:
\begin{itemize}
	\item update the entire project to Android Studio and Gradle;
	\item split the 3 demo applications in different projects;
	\item put SPFShared and SPFLib into a Maven repository to be able to import in all project what you want to create SPF's applications;
	\item update all GUIs to Material Design and in particular using the necessary support library by Google
	\item completely remove AllJoyn/AllSeen middleware and replace it with a pure and complete Wi-Fi Direct middleware
	\item improve Wi-Fi Direct Middleware's architecture and sourcecode improving the reliability
	\item support Wi-Fi Direct groups made by 2 or more devices
\end{itemize}

All projects are available on GitHub licensed under LGPLv3 and I also included Travis Continuous Integration to auto compile the source code at every \textsf{git push} command, in particular to check the different releases.
The official repositories are:
\begin{enumerate}
	\item https://github.com/deib-polimi/SPF2
	\item https://github.com/deib-polimi/SPF2CouponingProviderDemo
	\item https://github.com/deib-polimi/SPF2CouponingClientDemo
	\item https://github.com/deib-polimi/SPF2ChatDemo
\end{enumerate}
The first one is the main project, with also SPFShared and SPFLib. In this project all sub-modules are connected using normal gradle module's dependency.
I.e. there is a settings.gradle file where are declared all submodules and
in every local module's build.gradle are specified the dependency from other sub module.
For example, this is the settings.gradle
include ':sPFShared'
include ':sPFFramework'
include ':sPFWFDMid'
include ':sPFLib'
include ':sPFApp'

and this one is the build.gradle in sPFFramework sub module:
dependencies {
    compile project(':sPFShared')
    compile 'com.google.code.gson:gson:2.4'
    compile 'com.android.support:support-v4:23.1.0'
    provided 'org.projectlombok:lombok:1.16.6'
}

As you can see there are local module's dependencies like gson and support-v4, a provided dependency like lombok  and finally the dependency of sPFShared module, not online like the others, but local into the Android Studio Project. In fact, all the dependencies in SPFApp are local because it's the place where you can develop their. In external apps, like SPFCouponing and SPFChat, all dependencies are remote, using the deployed version of SPFShared and SPFLibs. This is the correct way to manege dependencies for SPF.
A very big improvement should be the total separation of SPFApp from sPFFramework and all other sub modules. But this is another topics and i will show more details and hints at the end of this document, because should be the main objective for the next major release.


