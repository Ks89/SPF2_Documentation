\chapter{New Material Design GUI}
\label{gui}

In this short chapter I'll show the main improvements to the GUI.
Mainly, I used all available elements into the support library and I replaced the \emph{Holo Theme} with \textsf{Theme.AppCompat.Light.NoActionBar} to mimic the \emph{Material Theme}. Also, when necessary, I used \textsf{Compat} elements, like \textsf{CompatButton} and so on to show the same element in different Android versions. When possibile, I replaced unofficial libraries or old components with newer versions from \emph{Google Design support library}, for example to create tab layouts. 

\begin{figure}[thpb]
\centering
\begin{minipage}[b]{0.4\textwidth}
	\centering
	\includegraphics[scale=0.1]{./images/chap3/drawer.png}
	\caption{Drawer}
	\label{fig:drawer}
\end{minipage}
\hfill
\begin{minipage}[b]{0.4\textwidth}
	\centering
	\includegraphics[scale=0.1]{./images/chap3/bottom_iconics.png}
	\caption{Buttons with Iconics}
	\label{fig:buttom-iconics}
\end{minipage}	
\end{figure}	

I applied all these updates to all SPF's apps. Also, I implemented the \textsf{Material Drawer} using a custom unofficial library available on GitHub, created by Mike Penz. The idea was to build a cool Material Drawer very quickly, without to implement manually some annoying features. The result is in Figure \ref{fig:drawer}. In addition, I used \textsf{Iconics} library to add vector icons to buttons, toolbar and to the drawer's items, as in Figure \ref{fig:buttom-iconics}.

\begin{figure}[thpb]
\centering
\begin{minipage}[b]{0.4\textwidth}
	\centering
	\includegraphics[width=\textwidth]{./images/chap3/header.png}
	\caption{Header}
	\label{fig:header}
\end{minipage}
\hfill
\begin{minipage}[b]{0.4\textwidth}
	\centering
	\includegraphics[width=\textwidth]{./images/chap3/switches.png}
	\caption{Drawer's switches}
	\label{fig:switches}
\end{minipage}	
\end{figure}


\begin{figure}[thpb]
	\centering
	\includegraphics[scale=0.15]{./images/chap3/about_fragment.png}
	\caption{About Fragment}
	\label{fig:about}
\end{figure}	


As you can see, i added a \emph{Header} with an image and the profile photo (Figure \ref{fig:header}) and a \emph{Sticky Footer} into the \emph{Drawer}, with switches (Figure \ref{fig:switches}). Also, I added two new items: \textsf{Group Infos} and \textsf{About}. The first one selects a fragment with the available services and the connected devices. This is not necessary to users, but it very useful during development to understand if a device is still connected or if there are problems related to Wi-Fi Direct. The second one is a simple about page of SPF, created with a library that compose this Activity scanning the libraries and dependencies used inside this application, as in Figure \ref{fig:about}.

\begin{figure}[thpb]
	\centering
	\includegraphics[scale=0.15]{./images/chap3/tablet1.png}
	\caption{Multi-pane layout}
	\label{fig:tablet1}
\end{figure}	


After this quick explanation, I want to explain one of these features, the 	\emph{Material Drawer}. In particular, I want to show how to reuse the same \emph{Drawer} for smartphones and tablets to create a \emph{Multi-pane layout}, as in Figure \ref{fig:tablet1}.

In Listing \ref{lst:multipane} there is a shorter version of the code that I used to create the drawer. All the method's invocations are chained, because every method returns the same object, a \textsf{DrawerBuilder}'s instance.


\begin{lstlisting}[caption={DrawerBuilder creations},label=lst:drawerbuilder, language=Java]
drawerBuilder = new DrawerBuilder()
  .withActivity(this)
  .withAccountHeader(headerResult)
  .withHasStableIds(true)
  .withToolbar(toolbar)
  .addDrawerItems(
    new PrimaryDrawerItem().withName(mSectionNames[0])
      .withIdentifier(0).withIcon(FontAwesome.Icon.faw_user),
    	(...)
  )
  .withOnDrawerItemClickListener(drawerItemClickListener)
  .withSavedInstance(savedInstanceState);
\end{lstlisting}


And finally, in Listing \ref{lst:multipane}, I added the code to reuse the \textsf{MaterialDrawer} to create a \emph{Multi-pane layout}. The trick is to add the view obtained from the \textsf{Drawer} to the \textsf{viewGroup} related to the layout's element, a simple \textsf{FrameLayout} that I called \textsf{nav\_tablet}. Obviously, you need two layouts, one for smartphones and one for tablet. In the first case you must use a \textsf{DrawerLayout}, in the second one you could simply use a \textsf{FrameLayout}.


\begin{lstlisting}[caption={Reuse Drawer for Multi-pane layout},label=lst:multipane, language=Java]
if (tabletSize) {
  //if tablet create a multipane layout
  drawer = drawerBuilder.buildView();
  ((ViewGroup) findViewById(R.id.nav_tablet))
    .addView(drawer.getSlider());
} else {
  //on smartphones i want to show the hamburger icon
  if (getSupportActionBar() != null) {
    getSupportActionBar().setDisplayHomeAsUpEnabled(false);
  }
  drawerBuilder.withActionBarDrawerToggle(true);
  drawerBuilder.withActionBarDrawerToggleAnimated(true);
  drawer = drawerBuilder.build();
  drawer.getActionBarDrawerToggle()
    .setDrawerIndicatorEnabled(true);
}
\end{lstlisting}


